\lstinputlisting[
    name=Задание 1
]{_input/1/1.md}

Создаём файл командой \textbf{copy con имяФайла}.

Заполняем файл текстом. Как заполнили, нажимаем \textbf{Ctrl} + \textbf{Z} (символ конца), затем \textbf{Enter}.

Проверяем содержимое файла командой \textbf{type имяФайла}.

\lstinputlisting[
    name=Создание файла test3.txt
]{_input/1/copy_con_text3.txt.txt}

Создаём каталог с помощью команды \textbf{mkdir имяКаталога}.

Удостовериваемся о создании командой \textbf{dir}.

\lstinputlisting[
    name=Создание каталога LAB
]{_input/1/mkdir_LAB.txt}

\lstinputlisting[
    name=Создание файлов file1 file2 file3 file4.txt file5.txt file6.txt
]{_input/1/copy_con_file1_file2_file3.txt}

Выводим атрибуты файла командой \textbf{attrib имяФайла}.

\lstinputlisting[
    name=Вывод атрибутов test3.txt
]{_input/1/attrib_test3.txt.txt}

\textbf{+r} устанавливает атрибут "Только чтение".

\lstinputlisting[
    name=Назначение атрибута "Только чтение" test3.txt
]{_input/1/attrib_+r_test3.txt.txt}

\textbf{-r} снимает атрибут "Только чтение".

\lstinputlisting[
    name=Удаление атрибута "Только чтение" всех файлов txt из каталога LAB
]{_input/1/attrib_-r_LAB_x.txt.txt}

\lstinputlisting[
    name=Отменяем атрибута "Только чтение" для всех файлов (*) из каталога LAB
]{_input/1/attrib_-r_LAB_x.txt}

\lstinputlisting[
    name=Установка атрибута "Архивный" для всех файлов на диске r
]{_input/1/attrib_s_+a.txt}
